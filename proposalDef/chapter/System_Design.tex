\chapter{System Design}
\section{Requirement Analysis}
        \subsection{Functional Requirements}
            These are specifications that describe the fundamental capabilities and behaviors a system or product must exhibit to meet the users' needs and achieve its intended purpose. 

            \subsubsection{Tokenization}
            The system should be able to tokenize Nepali text into individual words or subword units, considering the complexities of the Nepali script.

            \subsubsection{Embedding Generation}
            Generate dense vector representations (embeddings) for each word or subword unit in the Nepali vocabulary. These embeddings should capture semantic relationships between words.

            \subsubsection{Embedding Lookup}
            Allow users to retrieve the embedding vector for any given word or subword unit in the Nepali vocabulary.

            \subsubsection{Similarity Calculation}
            Calculate semantic similarity between words or subword units based on their embedding vectors. Users should be able to compare words and get similarity scores.

        \subsection{Non-Functional Requirements}
            These are the characteristics and qualities that describe how a system should behave and perform.

            \subsubsection{Usability}            
            The system should have a user-friendly interface or API that allows users to easily interact with word embeddings without requiring deep technical knowledge.

            \subsubsection{Reliability}
            The system aims to be highly available, with minimal downtime, to ensure users can access word embedding functionalities when needed.

            \subsubsection{Interoperability}
            The system ensures compatibility with various operating systems, programming languages, and NLP frameworks to facilitate integration with existing systems and workflows.

            \subsubsection{Maintainability}            
            The system with a modular architecture will enable easy maintenance, updates, and future enhancements.


        % \newpage        
        \subsection{Feasibility Study}
            The following points describes the feasibility of the project.

            \subsubsection{Economic Feasibility}
                The total expenditure of the project is just computational power. The computational resources can be fulfilled with the help of college. Therefore, the project is economically feasible.

            \subsubsection{Technical Feasibility}
                While existing datasets on this topic are available, they are insufficient. But, by using the abundance of Nepali news articles, books, and literature accessible online can augment our corpus significantly through web scraping.

            \subsubsection{Operational Feasibility}
                The operational processes, including web-scraping and model training, are well-defined and can be efficiently carried out by the project team. Additionally, the project aligns with the existing technical infrastructure and capabilities, making it operationally feasible.
%\textbf{Note: All the diagram is not mandatory..select according to the type of project you are going to do.}