\chapter{Implementation Plan}
\section{Schedule(Gantt Chart)}
\vspace{2em}
        \begin{figure}[htbp]
            \hspace*{-1cm} % Adjust the value as needed
            \begin{ganttchart}[
                hgrid,
                vgrid,
                x unit=2.5em,
                y unit chart=4em,
                time slot format=isodate-yearmonth,
                time slot unit=month,
                title height=1,
                bar/.append style={fill=blue!40},
                bar height=0.7,
                bar label node/.append style={align=right, text width=7em},
                group height=0.6,
                group peaks height=0.2,
                group peaks tip position=0,
                group/.append style={draw=black, fill=green!50},
                group label font=\small
                ]{2024-04}{2025-03}
                \gantttitlecalendar{year} \\
                \gantttitlecalendar{month} \\
                \ganttbar{Requirement Analysis}{2024-04}{2024-05} \\
                \ganttbar{Feasibility Study}{2024-04}{2024-05} \\
                \ganttbar{Software Design}{2024-05}{2024-06} \\
                \ganttbar{Implementation}{2024-06}{2025-01} \\
                \ganttbar{Testing}{2025-01}{2025-02} \\
                \ganttbar{Research}{2024-04}{2025-01}\\
                \ganttbar{Documentation}{2024-04}{2025-03}\\
            \end{ganttchart}
            \caption{Gantt Chart}
        \end{figure}
\newpage
\section{Hardware and Software requirements}
    \subsection{Software Requirement}
    This project requires following softwares:
    \subsubsection*{Python} 
    Python is our primary language for this project, chosen for its versatility and simplicity. All machine learning frameworks are imported using Python, leveraging its dominant position in data science and machine learning. This allows us to seamlessly integrate powerful libraries like TensorFlow and PyTorch for efficient model development. It also provides essential libraries like Scikit-learn, Numpy, Pandas, etc.

    \subsubsection*{Pytorch}
    Pytorch is a an open-source machine learning framework. It provides a flexible and dynamic computational graph, which allows for easy experimentation and rapid development of deep learning models. 

    \subsubsection*{Natural Language Toolkit}
    Natural Language Toolkit (NLTK) is a free and open source Python library for natural language processing. NLTK provides stemming, lowercase, categorization, tokenization, spell check, lemmatization, and semantic reasoning text processing packages. It gives access to lexical resources like WordNet.


    \subsection{Hardware Requirement}
    \subsubsection*{Memory}
    Word embedding models like BERT need a lot of memory to work. Training BERT needs several gigabytes of RAM, maybe even more if the model is really big. 

    \subsubsection*{Processing Power}
    Training these models takes a lot of computing power, especially when working with big datasets or complex models. Having a strong computer with a good CPU or GPU can help speed up the training process. GPUs are especially useful for this because it can handle lots of tasks at once.

    \subsubsection*{Storage Space}
    Storing all the data and models can take up a lot of space. Adequate disk space is required to store necessary files, including the dataset, the models, and any other files related to the project.

% \section{Cost Estimation}

